\documentclass[12pt,letterpaper]{article}

\usepackage{amsmath} %Used for non-numbered formulas
\usepackage{graphicx} %Used for figures
\usepackage{subcaption} %Used for putting figures next to each other
\usepackage{setspace} %Double-spacing, etc
\usepackage{amssymb} %Some special symbols like the slanted >= sign

%The following is so that new sections are placed on another page if they are too close to the bottom
	\usepackage{needspace}
	\usepackage{xpatch}
	\xpretocmd{\section}{\clearpage}{}{} %New section = new page
	\xpretocmd{\subsection}{\needspace{15\baselineskip}}{}{}
	\xpretocmd{\subsubsection}{\needspace{15\baselineskip}}{}{}
	
\usepackage[hidelinks]{hyperref} %clickable TOC
\hypersetup{
	allcolors=false
}
	

%equation vs equation*
%align vs align* (uses &)


\title{"The Collatz Playground"}
\author{Frank Rodriguez}
\date{}

\spacing{1.1}

\begin{document}
	\pagenumbering{gobble}
	
	\maketitle
	\newpage
	
	\pagenumbering{roman}
	
	\tableofcontents
	\newpage
	
	\section*{Preface}
	
		\paragraph{} Learning higher mathematics can be an intimidating prospect. The leap from methodical calculations done in classrooms to the abstract thought required for "real" math can be quite large, especially without experience in putting theory into practice.
		 
		\paragraph{} Upon learning of a statement known as the \textit{Collatz Conjecture}, I was smitten. The behavior is wonderfully complex. Stumping mathematicians of all skill levels for nearly a century, it is suspected of being completely impossible to solve. However, the entry point was incredibly simple, so much so that an elementary schooler could understand the concept!  I could not help but play with it.
		
		\paragraph{} After spending a considerable amount of time on the problem, I am nowhere near solving it; that is not my true goal. I believe this is a lovely way to learn of the side of math rarely covered in grade school: proofs. Throughout my foray, I have written many small proofs as tools for proving more complex behavior, and this is the first time I have had this level of experience with such concepts. Collatz's playground has proven to be an enjoyable place indeed...
		\\ \\ \\
		
		\paragraph{} If anything comes of this, let it be that my fun inspires you to find your own playground. \\ \\ -Frank Rodriguez
		
		
		\newpage
		
	\pagenumbering{arabic}
	
	\section{Diving In}
	
		\subsection{A Brief Introduction}
		
			\subsubsection{What is the Collatz Conjecture?}
			
				\paragraph{} The conjecture itself is extremely simple: Pick any integer (a whole number) larger than 1. If this number is even, divide it by 2. If it is odd, multiply it by 3, then add 1. The resulting number is your new number. Now, perform the same decision-making process on it again. 
				
				\paragraph{} The Collatz Conjecture states that, no matter which number you chose (so long as it is an integer 1 or larger), you will eventually hit 1.
				
				\begin{figure}[h] 
					\begin{displaymath} 
					C(n) = 
					\begin{cases}
					\frac{n}{2} &\text{if n is even}\\
					3n+1 &\text{if n is odd}
					\end{cases}
					\end{displaymath}
					
					\caption{The Collatz Function}
					\label{function:collatz}
				\end{figure}
				
				
				\paragraph{} This has been tested with a very large set of numbers,\footnote{I believe past $2\strut^{60}$.} and they all reach 1 eventually. This indicates that the conjecture is \textit{probably} true, but to this day, no one has been able to create a formal proof, due in no part to a lack of effort! The problem posed is simply extremely difficult. How does one begin to tackle a question such as this?
				
			\subsubsection{Find Something to Find}
			
				\paragraph{} If we assume the conjecture to be \textit{true}, then proving it will require some impressively abstract thought. We would be searching more for pure \textit{ideas}, rather than any sort of \textit{numerical} values. While undoubtedly interesting, I decided to try tackling this problem from the other end: what if it is false?
				
				\paragraph{} If the Collatz Conjecture is false, that gives us something to look for: a number that disproves the conjecture by never reaching 1, no matter how many iterations of the Collatz Function\footnote{I'll refer to the Collatz Function often: know that it refers to the piecewise function defined in figure \ref{function:collatz}} we pass it through.
				
		\subsection{The Magic Number}
		
			\subsubsection{The Identity}
				
				\paragraph{} The Collatz Conjecture is really only concerned with numbers $\geqslant 1$\footnote{This can be thought of as the \emph{type} of number we are working with. Types are incredibly important, as they offer implicit information for a problem. In the case of the Collatz Conjecture, we are discussing integers 1 and larger the overwhelming majority of the time.}, so there is indeed an established lower bound. It is quite clear that $1 = 1$, so it is of no concern. 2 follows an easy path, where $C(2) = 1$. We could keep going, but instead, let us think about numbers which might disprove the Collatz Conjecture by never approaching 1.
				
				\paragraph{} The first important realization is that there cannot exist a \textit{single} number which approaches 1 -- it must be a part of a larger chain. In other words, for any $n \geqslant 1$, $C(n) \neq n$\footnote{A common theme seen is how mathematical syntax is capable of succinctly and explicitly expressing ideas which would be incredibly verbose in English. This statement can be read as follows: For any $n$ greater than or equal to 1, putting $n$ into the Collatz Function will never return our original $n$ -- it must be something different. It is a very good idea to become comfortable with statements such as these.}. A repeating chain can be seen when starting from the number 1:
				
				\begin{figure}[h]
					\begin{align*}
					C(1) &= 4\\
					C(4) &= 2\\
					C(2) &= 1\\
					C(1) &= 4\\
					\text{...}
					\end{align*}
					
					\caption{1's repeating chain}
					\label{chain:1}
				\end{figure}
			
			\paragraph{} If there are multiple numbers which disprove the conjecture, then logically one of them must be closer to 1 than all the rest. Let us call it $n$. Since $n$ is the lowest disproving number, then $C(n) > n$ must be true. The only way this is possible is if $n$ is odd, since an even number would get divided by 2, thus becoming smaller.
			
			\paragraph{} To summarize, if a disproving number exists, then \textit{multiple} disproving numbers must exist. If multiple exist, then one must be the lowest one. If one is the lowest one, then it must be odd. That is enough information to begin reasoning about the number's behavior, as well as the behavior of the overall chain it belongs to. What does it take for a number to never touch 1?
			
		\subsubsection{The Needle in the Hay} \label{sss:needle}
				
			\paragraph{} Now it is time to put legitimate effort into considering how our number, $n$, must behave. At the very least, how \textit{might} it behave? It is very important to remember that we are defining $n$ as "the lowest number which does not reach 1, regardless of how many times it is run through the Collatz Function." Therefore, at no point can $n$'s chain include a number \textit{less} than $n$.
			
			\paragraph{} This is why we know that $n$ must be odd, but what about $C(n)$? $C(n) = 3n + 1$. Is this even, or odd? We know that $n$ is odd, and it is being multiplied by $3$, another odd number. Is there any guarantee about an odd number multiplied by another odd number? Actually, yes. When you multiply two integers together, it can be thought of as combining their factors:
			
			\begin{figure}[h]
				\begin{align*}
					15 &= 3 \cdot 5\\
					9 &= 3 \cdot 3\\
					15 \cdot 9 &= (3 \cdot 5) \cdot (3 \cdot 3)\\
					&= 3 \cdot 5 \cdot 3 \cdot 3 \\
					&= 3\strut^3 \cdot 5\\
					&= 135
				\end{align*}
				
				\caption{Multiplication is the combination of factors}
				\label{fig:multiplication}
			\end{figure}
		
			\paragraph{} By definition, an odd number is any whole number which does not have 2 as a factor. If a number had 2 as a factor, you could rewrite it as 2 multiplied by the rest of the factors, or $2(x)$, meaning it would be divisible by 2 evenly. Since multiplication is the \textit{combination}, not \textit{creation} of factors, if two numbers do not have 2 as a factor and are multiplied together, then the resulting number also must not have 2 as a factor. Similarly, it can be shown that any integer multiplied by an even integer must be even.
			
			\paragraph{} Thus, $3n$ must be odd. Since odds and evens alternate, an odd $+ 1$ must be an even, so $3n + 1$ must be even. Therefore, $C(3n + 1) = \frac{3n + 1}{2}$. Now, is this new number even or odd? It is possible for a number to be divisible by 2 twice. For example, $\frac{12}{2} = 6$, and $\frac{6}{2} = 3$, but at the same time it is not guaranteed. With just the information "$x$ is even", it is impossible to answer the question "is $\frac{x}{2}$ even?"
			
			\paragraph{} Luckily, that is not all of the information we have. If our new number was even, then $C(\frac{3n+1}{2}) = \frac{\frac{3n+1}{2}}{2} = \frac{3n+1}{4}$. This number would be less than $n$ for any $n > 1$:
			
			\begin{figure}[h]
				\begin{align*}
					\frac{3n+1}{4} &< n\\
					3n+1 &< 4n\\
					1 &< n\\
					n &> 1
				\end{align*}
			\end{figure}
		
			\paragraph{} Remember that we cannot have a term in the sequence that is less than $n$. Therefore, $n$ cannot be even, so it must be odd. This means that the next term \textit{must} be: $C(\frac{3n+1}{2}) = 3\left( \frac{3n + 1}{2} \right) + 1= \frac{9n+5}{2}$. If this is even, we would have $C(\frac{9n+5}{2}) = \frac{9n+5}{4}$, which is still greater than $n$, so it definitely could be even here. It could also be odd, because $3\left(\frac{9n+5}{2}\right) + 1$ is still greater than $n$.
			
			\paragraph{} Since both outcomes are valid, so ends the chain of guaranteed behavior. I assert that, assuming a number $n$ exists which is the lowest disproving number of the Collatz Conjecture, then $n$'s chain must start as such:
			
			\setcounter{equation}{0}
			\begin{figure}[h]
				\begin{align}
					&n\\
					&3n+1\\
					&\frac{3n+1}{2}\\
					&\frac{9n+5}{2}
				\end{align}
				
				\caption{The hypothetical beginning of the disproving chain}
				\label{fig:disproofChainStart}
			\end{figure}
		
		\newpage %aligns above figure
		
		\subsubsection{An Infinite Dance} \label{sss:infinite}
		
			\paragraph{} Although the pattern past that third term cannot be known for certain, it can be speculated. What does it look like to not approach 1? Well, a number which approaches infinity would do the trick. What would such a chain look like?
			
			\paragraph{} The four terms shown in figure \ref{fig:disproofChainStart} have the property that the last term ($\frac{9n+5}{2}$) is larger than the first ($n$). Since it is proven for a general $n$, a sort of "freebie" is that we can stack these in an infinitely long chain, with each chunk proven to be getting successively larger thanks to the work done in \ref{sss:needle}. After all, this is why these numbers were chosen, precisely because of this property.
			
			\paragraph{} Therefore, a simple chain which tends towards infinity can be seen as a chain of numbers, where the first term is odd (our $n$), the next is even, the next is odd, then even, then odd, then even... repeated infinitely. This produces the infinite sequence:
			
			\setcounter{equation}{0}
			\begin{figure}[h]
				\begin{align}
					\text{Start: }& n\\
					\text{Odd: }& C(n) = 3n+1\\
					\text{Even: }& C(3n+1) = \frac{3n+1}{2}\\
					\text{Odd: }& C\left(\frac{3n+1}{2}\right) = 3 \left(\frac{3n+1}{2}\right) + 1\\
					\text{Even: }& C\left(3 \left(\frac{3n+1}{2}\right) + 1\right) = \frac{3 \left(\frac{3n+1}{2}\right) + 1}{2}\\
					\text{Odd: }& C\left(\frac{3 \left(\frac{3n+1}{2}\right) + 1}{2}\right) = 3 \left(\frac{3 \left(\frac{3n+1}{2}\right) + 1}{2}\right) + 1\\
					\text{Even: }& C\left(3 \left(\frac{3 \left(\frac{3n+1}{2}\right) + 1}{2}\right) + 1\right) = \frac{3 \left(\frac{3 \left(\frac{3n+1}{2}\right) + 1}{2}\right) + 1}{2}\\
					\text{Odd: }& C\left(\frac{3 \left(\frac{3 \left(\frac{3n+1}{2}\right) + 1}{2}\right) + 1}{2}\right) = 3\left(\frac{3 \left(\frac{3 \left(\frac{3n+1}{2}\right) + 1}{2}\right) + 1}{2}\right) + 1
				\end{align}
				
				\caption{The beginning of the infinite chain, unsimplified}
				\label{chain:unsimplified}
			\end{figure}
			
			\newpage %aligns the above figure
			
			\paragraph{} Looking at it unedited shows the structure, the back and forth between evens and odds. However, actually working with it in this form is quite difficult, so a simplification is in order, seen in figure \ref{chain:simplified} below.
			
			\setcounter{equation}{0}
			\begin{figure}[h]
				\begin{align}
					&n\\
					&3n+1\\
					&\frac{3n+1}{2}\\
					&\frac{9n+5}{2}\\
					&\frac{9n+5}{4}\\
					&\frac{27n+19}{4}\\
					&\frac{27n+19}{8}\\
					&\frac{81n+65}{8}
				\end{align}
				
				\caption{The beginning of the infinite chain, simplified}
				\label{chain:simplified}
			\end{figure}
		
			\newpage %aligns above figure
		
			\paragraph{} Now, some patterns are much more evident. First of all, each term in the sequence is representable as $\frac{An + B}{C}$, a neat little package to parse. While we could certainly calculate very many of these by hand, it would serve our purposes much better to come up with a generalized function $f$ which calculates these numbers for us. Say we set $n$ to 4 and wanted to see what would happen after 5 iterations in our function, we could calculate $f(12, 5)$.
			
			\paragraph{} We know that 4 is \textit{not} our magic number, because $C(4) = 2$ and $C(2) = 1$. Our new function $f$ would treat 4 as an odd number, then an even number, then an odd number, etc. This is against the rules of the original Collatz Function, which are designed to ensure that the output of the function is always another integer. Breaking these rules will inevitably result in trying to divide an odd number by 2, resulting in a fraction rather than an integer. 
			
			\paragraph{} This can be shown as such: If some number $x$ is even, then attempting to perform $3x+1$ on it will result in an odd number. If the next action is to divide it by 2, a fraction has been forced. Alternatively, if $x$ is already odd and is immediately divided by 2, another fraction has been forced. Therefore, diverging from the rules forces a fraction.
			
			\paragraph{} This is a powerful tool: it tells us that, if we can find a number $n$ which can be plugged into $f$ and iterated upon infinitely (thus tending towards infinity, since $f$ is the function form of our odd-even-odd-even infinite chain), then this number $n$ is our magic disproving number.
			
			\paragraph{} However, before we can begin finding it, we need to define $f$...
			
	\subsection{The Generator Triplet}
	
		\subsubsection{Groundwork}
		
			\paragraph{} Before we begin, it would behoove us to clearly define exactly what $f$ will be.
			
			\paragraph{} $f$ will be the function form of the infinite chain given in figure \ref{chain:simplified}. It will take an input $n$ representing the number to work on, and an input $s$ representing which step of the sequence to return. Thus, $f(n, 2) = \frac{3n+1}{2}$, as $n$ would be 0 steps, $3n+1$ would be 1 step, and $\frac{3n+1}{2}$ would be 2 steps.
			
			\paragraph{} It is clear that the skeletal form of $f$ takes the form $f(n, s) = \frac{An + B}{C}$. The next step is solving for $A$, $B$, and $C$. Here, "solve" means something different than we are used to. This is not a system where $A$, $B$, and $C$ have discrete values. We will not find that, for example, $A = 5$. Rather, they can be thought of as functions in their own right, since they will change with the input according to some rules. As far as which rules, that is the heart of what we are solving.
			
			\paragraph{} By hand, we can work out a handful of numbers from the respective $ABC$ chains, in order to facilitate finding the rules. The key is to look for patterns, and figure out how to mathematically encapsulate them:
			
			\begin{figure}[h]
				\centering
				\begin{tabular}{l | c | c | r}
					s & A & B & C \\
					0 & 1 & 0 & 1 \\
					1 & 3 & 1 & 1 \\
					2 & 3 & 1 & 2 \\
					3 & 9 & 5 & 2 \\
					4 & 9 & 5 & 4 \\
					5 & 27 & 19 & 4 \\
					6 & 27 & 19 & 8 \\
					7 & 81 & 65 & 8 \\
					8 & 81 & 65 & 16 \\
					9 & 243 & 211 & 16 \\
					10 & 243 & 211 & 32 \\
				\end{tabular}
			
				\caption{The expected first 11 outputs of $A$, $B$, and $C$}
				\label{table:sabc}
			\end{figure}
		
			\newpage %aligns above figure
			
			\paragraph{} Note that we do not need to solve for $s$, it is simply our counter for which step we are on.
			
			\paragraph{} To me, this is immensely enjoyable due to its puzzle aspects. One or two of these chains have very clear patters, even now you might see the rules. Finding precise descriptions of the behaviors of such constructs can be very challenging indeed!
			
		\subsubsection{Solving for C} \label{sss:c}
			
			\paragraph{} There is no reason to work on these in order. In fact, the one which appears to be the easiest is $C$, so starting there seems like a good idea. The terms in the $C$ chain are clearly powers of 2, but the solution is not so simple as to declare $C_s = 2\strut^s$. A test shows the error: $C_3 = 2\strut^3 = 8$, when figure \ref{table:sabc} shows that $C_3$ instead equals 2. 
			
			\paragraph{} Yes, there is a power-of-two pattern occuring for $C$, but it seems to be staggered; only every other term is a power of two, the others are copies of the previous term. How can this be, for a function to be directly correlated to an input, but only sometimes? It certainly struck me as a curious phenomenon.
			
			\paragraph{} Luckily, there is a tool which can be used to help with this: the \textit{floor} ($\lfloor \rfloor$) and \textit{ceiling} ($\lceil \rceil$) operators. These may be new to you, but they are actually quite simple. The floor operator simply rounds a number down to the nearest integer. For example, $\lfloor \frac{1}{2} \rfloor$ is 0. If it is used on a number which is already an integer, nothing happens. $\lfloor 10 \rfloor$ is just 10. Ceiling is identical, except it rounds \textit{up}. $\lceil 10.1 \rceil$ is 11.
			
			\paragraph{} Let us quickly look back at $C$'s chain. The first term is equal to $2\strut^0$, the second is $2\strut^0$ again, the third is $2\strut^1$, the fourth is $2\strut^1$, the fifth is $2\strut^2$, and so on. We can take a step towards solving for $C$ by declaring that $C_s = 2\strut^x$. Now we need to fill in the blank given by $x$.
			
			\paragraph{} The previous paragraph shows that the exponent which the 2 is raised to follows yet another pattern: 0, 0, 1, 1, 2, 2, 3, 3, 4, 4, and so on. Let us name this $P_1$, for pattern 1. Perhaps this is easier to tackle with our new friends, the floor and ceiling operators?
			
			\paragraph{} As $s$ increases, it will follow the pattern 0, 1, 2, 3, 4, 5, 6, 7, etc. This will be referred to as $P_2$. We wish to \textit{transform} $P_2$ into $P_1$ using some function, so that this function can be placed into the $x$ in $C$. I have found that the best approach is to take it one number at a time.
			
			\paragraph{} First, find a function which transforms the first element of $P_2$ (which we can refer to as $P_2(1)$) into $P_1(1)$. Then, see if this works for $P_2(2) \rightarrow P_1(2)$. If it does not, find a function which works for both $P_2(2) \rightarrow P_1(2)$ \textit{and} $P_2(1) \rightarrow P_1(1)$. If it does, keep checking until you have found reason to believe that your function works for the entire set $P_2(s) \rightarrow P_1(s)$. While the first guess is usually wrong, refining it so that it works for more and more transformations along your two sets will result in a function closer and closer to the answer (hopefully!).
			
			\paragraph{} Back to our tangible $P_1$ and $P_2$. In this case, $P_1(1)$ already equals $P_2(1)$, so our temporary function $t$ might just be $t(n) = n$. Does this definition of $t$ work for $P_2(2) \rightarrow P_1(2)$? It absolutely does not, $1 \neq 0$. Now, how can we define $t$ such that $t(1) = 0$ and $t(0) = 0$? One way might be $t(s) = 0$, and this certainly works. But what of $P_2(3) \rightarrow P_2(3)$? No, $0 \neq 1$.
			
			\paragraph{} Now we need to define $t$ such that $t(2) = 1$ and $t(1) = 0$ and $t(0) = 0$. One might be tempted to use a piecewise function here, and it certainly will work, but be wary of using piecewise functions as a crutch. Not only can they be quite difficult to work with mathematically (many lend themselves to \textit{irreversible functions}: functions where knowing the output alone is not enough to discern the input), but they can be infinitely expanded, meaning we technically can come up with a rule for turning every element of $P_2$ into $P_1$ in isolation. That is, assuming we could come up with infinite rules. We clearly cannot, and it is required for this, so there must be another way.
			
			\paragraph{} Feel free to stop here for a while and work on it for a little bit. I, however, will carry on.
			
			\paragraph{} If we floor each element in $P_2$, we will not actually do anything, since each element is already an integer. Therefore, we have to manipulate it in some way before flooring it. We could take a little bit off, say, 0.1, and that would make each term collapse to one less than it. If we define $t$ as $t(s) = \lfloor s - 0.1 \rfloor$, then $t(1) = 0$, just as expected. However, $t(0)$ then equals $-1$.
			
			\paragraph{} In order to respect that 0, we need an operation which will not change it. Addition and subtraction will, but multiplication and division are two operations which will not. Multiplying our number, though, will result in a larger one, though. A property we can see is that $P_1(s) \leqslant P_2(s)$. Therefore, division would be better suited.
			
			\paragraph{} Defining $t$ as $t(s) = \lfloor \frac{s}{2} \rfloor$ would produce interesting results. $t(0)$ is still 0. $t(1)$ is also 0. $t(2)$ is now 1, and $t(3)$ is also 1. Continuing the tests seems to show that this definition of $t$ seems to correctly transform $P_2(s)$ into $P_1(s)$. Can we rationalize this behavior to try and prove that it will work for all infinite elements in $P_1$ and $P_2$?
			
			\paragraph{} Well... the number is supposed to change every other element in $P_1$. That is, it changes every 2 elements. Since we are flooring $\frac{s}{2}$, $n$ has to increase by 2 before the number changes! If we wanted it to change every third element, we could instead floor $\frac{s}{3}$. It checks out!
			
			\paragraph{} We no longer need $t$ as a function. Instead, we can insert its definition directly into $C$'s skeletal definition, giving us $C_s = 2\strut^{\lfloor \frac{s}{2} \rfloor}$. Testing this filled-in $C$ against the expected values in figure \ref{table:sabc} shows a match. $C$ is solved! We can now add to our skeletal definition of $f$:
			
			\begin{equation*}
				f(n, s) = \frac{An + B}{2\strut^{\lfloor \frac{s}{2} \rfloor}}
			\end{equation*}
			
		\subsubsection{Solving for A}
		
			\paragraph{} This should be smoother, since we now have experience with finding these rules. $A$ is the next chain which seems to have a very regular pattern. We can see that it goes from 3 to 9 to 27. A quick check confirms that 27 is indeed $3\strut^3$. While $C$ was a chain of powers of 2, $A$ seems to be a chain of powers of 3.
			
			\paragraph{} Therefore, we can start to outline $A$ as $A_s = 3\strut^x$. Careful observation shows that the patterns for exponents are only \textit{similar} to $C$'s -- they are not an exact match. Checking the table shows that the exponents will form a $P_1$ of 0, 1, 1, 2, 2, 3, 3, 4, 4, 5, 5.... Our $P_2$, however, is identical because $s$ follows the same pattern for all 3 generators: 0, 1, 2, 3, 4, 5, 6, 7, 8, 9, 10....
			
			\paragraph{} Now the question begins again: how do we define some function $t$ such that $t(P_2(s)) = t(P_1(s))$? Since $C$ and $A$ have such a similar pattern, we can copy $C$'s $t$ definition over and use it as our starting point: $t(s) = \lfloor \frac{s}{2} \rfloor$. Testing it, however, shows that it indeed does not work. This is expected, since the patterns are different.
			
			\paragraph{} Still, note that while the patterns are different, they differ only mildly. In fact, $A$'s $P_1$ is identical to $C$'s $P_1$, except that it seems to be shifted to the left by one element -- it is missing that first 0. Therefore, it changes every other element, much like $C$, but its change is staggered to be one off from $C$. In other words, $A$ changes, then $C$, then $A$, then $C$, etc.
			
			\paragraph{} Now is a good time to try it on your own if you would like.
			
			
			\paragraph{} This opposite behavior makes me wonder if it simply uses the opposite function, ceiling, instead of floor: $t(s) = \lceil \frac{s}{2} \rceil$. Indeed, it seems to work beautifully, but we need to make sure.
			
			\paragraph{} Since $t$ is now rounding up rather than down, as soon as the term hits $\frac{1}{2}$, it will be set to 1 instead of staying at 0 as it does floor's case. However, $\lceil 0 \rceil = 0$, so there is no danger there. We are keeping $s$ divided by 2 for the same reasons as when solving for $C$, so that it changes every other time. However, for it to change every other time and also have that second element change, its "schedule" for changing must be flipped in precisely the way we are looking for. It seems like this checks out, as well!
			
			\paragraph{} Plugging $t$'s body back into our $A$ rough draft, we find $A_s = 3\strut^{\lceil \frac{s}{2} \rceil}$. Testing it against the expected values in figure \ref{table:sabc} shows that this is correct, and our reasoning about each component is sound. Therefore, we can consider $A$ solved, and plug it into $f$'s definition:
			
			\begin{equation*}
				f(n, s) = \frac{3\strut^{\lceil \frac{s}{2} \rceil} n + B}{2\strut^{\lfloor \frac{s}{2} \rfloor}}
			\end{equation*}
			
			\paragraph{} We are making quite the interesting function here...
			
		\newpage %align B section
		\subsubsection{The Mighty B}
		
			\paragraph{} Next is $B$. I assumed $B$ would also be something neat, such as a number raised to a conditional exponent, but that is not the case. In fact, looking at it may not reveal any patterns whatsoever. How do we extract any rules from this?
			
			\paragraph{} If you really stare at it, you might pick out some patterns. For example, it seems to grow much more quickly as $s$ increases, which is indicative of an exponential form. While this is helpful, for particularly difficult patterns, it is extremely beneficial to see what information can be pulled from outside of the system. In this case, we have more information than just the sequence: we know where this sequence came from, we can see the rules that generated it.
			
			\paragraph{} Refer to figure \ref{chain:unsimplified}, the unsimplified chain. Here is where seeing the raw structure is actually helpful. We know that, much like the others, $B$ seems to only change every other step. In the soup presented in figure \ref{chain:unsimplified}, which aspect of change seems to be associated with $B$?
			
			\paragraph{} Now is a good stopping point to work on the question yourself.
			
			\paragraph{} $B$ cannot be influenced by the even changes. In fact, it seems like \textit{only} $C$ is influenced by them. My reasoning is figure \ref{chain:simplified}: if the denominator had factors in the numerator, they could be factored out (note that this applies only to this general form of $\frac{An + B}{C}$). Since the denominator grows unabated, it must have no bearing on the numerator. Even if this was only true for the first few terms, it still shows that it must only be the odd terms which influence $B$, at least at the beginning.
			
			\paragraph{} Repeatedly calling $3n + 1$ on a number surely causes it to grow exponentially, as seen in $B$'s case. Indeed, each time a term grows, we can see that it grows by a factor larger than 3 of the previous one. In other words, when considering only the points of change, it seems to be true that $B_s > 3B_{s-1}$.
			
			\paragraph{} This hints at something nasty, though: if $B$ really does follow this behavior, then calculating $B_s$ would require calculating $B_{s-1}$, which would require $B_{s-2}$, which would require $B_{s-3}$, etc, all the way down to $B_0$. When calculating a term requires calculating the entire sequence before it, it is known as a \textit{recursive} sequence. For reasons which will become clear shortly, this is unfortunate.
			
			\paragraph{} Still, new information is always welcome, so let us try to find the rules of this recursive sequence. For now, we can ignore the repeating nature, and simply declare the sequence as 0, 1, 5, 19, 65, and so on. Why do we do this? 
			
			\paragraph{} Look back to $C$ in \ref{sss:c}. We created a skeletal definition of $2\strut^x$, taken from the observation that the sequence followed a general pattern of increasing powers of 2. Once we have that realization, all that needs to be changed is replacing $x$ with some modified $s$ to allow it to change every \textit{other} time. If it simply changes with $s$, then $2\strut^s$ would be our answer. These sort of problems can be solved in steps instead of all at once. 
			
			\paragraph{} In this case, first find the rule of increment that explains the changes. Then find the rule that describes how the increment is spaced out. They can be elegantly combined to find the total answer -- rarely is it a good idea to attempt to forcefully find the complete answer in one chunk.
			
			\paragraph{} So, let us look back to figure \ref{chain:unsimplified} to find this rule of increment. For reasons mentioned earlier, an even iteration does not directly influence $B$, so it must change on the odd iterations. Comparing figure \ref{chain:unsimplified} to the table of values given in figure \ref{table:sabc}, and even the simplified version given in figure \ref{chain:simplified} corroborates this. $B$ changes on the odds, with $A$.
			
			\paragraph{} The odd branch of the Collatz Function turns $n$ into $3n+1$. We can suspect that the $3$ is involved because $B_s > 3B_{s-1}$, but a relation ($>$) is not the same as the entire rule (which will instead have a $=$ sign). What about the $+1$? What effects does it have on $B$?
			
			\paragraph{} We can find an interesting truth through some arithmetic. Every odd term, we start with $\frac{An + B}{C}$ and end with $3\left(\frac{An + B}{C}\right) + 1$. This can be rewritten as $\frac{3(An + B)}{C} + 1$. We care about what happens to $B$, so let us distribute the 3: $\frac{3An + 3B}{C} + 1$. This proves that $B$ indeed gets multiplied by 3. 
			
			\paragraph{} Now for the 1. Recall that $1 = \frac{x}{x}$ so for this addition, we can perform it as such: $\frac{3An + 3B}{C} + \frac{C}{C} = \frac{3An + 3B + C}{C}$. $A$ is separate from $B$ because it is multiplied by our $n$, but $C$ and $B$ are both just integers, so they will end up being combined (as is seen in figure \ref{chain:simplified}).
			
			\paragraph{} Therefore, the equation directly related to how $B$ changes seems to be $3B + C$. Now we can turn this into a proper definition. Let us refer back to this table:
			
			\begin{figure*}[h]
				\centering
				\begin{tabular}{l | c | c | r}
					s & A & B & C \\
					0 & 1 & 0 & 1 \\
					1 & 3 & 1 & 1 \\
					2 & 3 & 1 & 2 \\
					3 & 9 & 5 & 2 \\
					4 & 9 & 5 & 4 \\
					5 & 27 & 19 & 4 \\
					6 & 27 & 19 & 8 \\
					7 & 81 & 65 & 8 \\
					8 & 81 & 65 & 16 \\
					9 & 243 & 211 & 16 \\
					10 & 243 & 211 & 32 \\
				\end{tabular}
				
				\captionsetup{list=no,format=hang}
				\caption*{The expected first 11 outputs of $A$, $B$, and $C$ (Copied from figure \ref{table:sabc})}
			\end{figure*}
		
			\paragraph{} Since we are finding a recursive sequence, our first term needs to be non-recursive so that the others have something to ground themselves with. If \textit{every} term referred to one before it, it would be impossible to extract any numbers without extra information because you would have to calculate infinite steps. In this case, we can see that the first term is $B_0 = 0$.
			
			\paragraph{} We are only looking at the points of change for now, so the sequence we are emulating is 0, 1, 5, 19, 65, 211, etc. We will also consider the $C$ values at these points: 1, 1, 2, 4, 8, 16, and so on. We have $B_0$ set, so we can remove the first terms from both of these chains. 
			
			\paragraph{} What happens if we follow the strictly follow the pattern we found earlier, $3B + C$? We would end up with $B_s = 3B_{s-1} + C_n$. The $s-1$ is there because it is "3 times the previous $B$." Does this work?
			
			\paragraph{} Let us find out. $B_0$ is just 0, no need to test anything. $B_1$ would be $3(0) + C_1 = 0 + 1 = 1$, and indeed it is. $B_2 = 3(1) + C_2 = 3 + 2 = 5$, again correct. In fact, they all match. More importantly, this makes sense because the behavior matches what we expected it to be when we found the $3B + C$ equation.
			
			\paragraph{} Now, how can we wrap this so that it only changes every other time? If we stop ignoring the constant terms in the sequence, we could explain them as such:
			
			\begin{figure}[h]
				\begin{align*}
					B_0 &= 0\\
					B_1 &= 3B_0 + C_1\\
					B_2 &= B_1\\
					B_3 &= 3B_2 + C_3\\
					B_4 &= B_3\\
					B_5 &= 3B_4 +C_5\\
					\text{...}&
				\end{align*}
				
				\caption{$B$'s behavior}
			\end{figure}
		
			\paragraph{} There seem to be two possibilities: Either $B_s$ is $B_{s-1}$ or $B_s$ is $3B_{s-1} + C_s$. Because there are only two possibilities, we can wrap this in a piecewise function for the purpose of conceptualizing what we are working with better. While probably a clerical nightmare, we shall name this function $B$ of course. It can be defined as follows:
			
			\begin{figure}[h]
				\begin{equation*}
					B(n) = \begin{cases}
						0 &\text{n = 0}\\
						B(s - 1) &s > 0 \text{ and s is even}\\
						3B(s - 1) + C_n &\text{s is odd}
					\end{cases}
				\end{equation*}
				
				\caption{$B$'s recursive definition}
				\label{function:bRecursive}
			\end{figure}
		
			\paragraph{} Therefore, our $f$ can be defined as follows, using the above $B$:
			
			\begin{figure}[h]
				\begin{equation*}
					f(n, s) = \frac{3\strut^{\lceil \frac{s}{2} \rceil} n + B(s)}{2\strut^{\lfloor \frac{s}{2} \rfloor}}
				\end{equation*}
				
				\caption{A definition of $f$}
			\end{figure}
		
			\paragraph{} This definition certainly works, but it feels... unsatisfying. We have built a definition for $B$ which relies on a recursive sequence and an external piecewise function, while we were able to create neat, compact rulesets for $A$ and $C$. The functionality of both $A$ and $C$ is also superior. It is easy to calculate what $A$ should be for the millionth $s$ in one smooth calculation, but doing the same for $B$ would require a million calculations. Is there a way to define its behavior more compactly?
			
			
		\subsubsection{The Almighty B}
			
			\paragraph{} In finding a definition for $B$, we have taken as much information as we can by working within the system. In order to see if a more succinct definition is possible, we must step back and take a wider look.
			
			\paragraph{} There is no real process for finding a solution like this. The best we can do is play with it and look for patterns, since patterns beget functions. In the spirit of the original $B$ solution, let us look at the structure of the recursive solution, without any simplification. As usual, we will only look at the points of increment. Also, how about we fill in the $C$ value for each point, so we have a number to look at. Finally, we will not simplify any exponents. We want this to be purely the raw structure:
			
			\begin{figure}[h]
				\begin{align*}
					&0 \\
					&3(0) + 2\strut^0\\
					&3(3(0) + 2\strut^0) + 2\strut^1\\
					&3(3(3(0) + 2\strut^0) + 2\strut^1) + 2\strut^2\\
					&3(3(3(3(0) + 2\strut^0) + 2\strut^1) + 2\strut^2) + 2\strut^3\\
					&3(3(3(3(3(0) + 2\strut^0) + 2\strut^1) + 2\strut^2) + 2\strut^3) + 2\strut^4\\
					&3(3(3(3(3(3(0) + 2\strut^0) + 2\strut^1) + 2\strut^2) + 2\strut^3) + 2\strut^4) + 2\strut^5
				\end{align*}
				
				\caption{$B$'s structure}
			\end{figure}
		
			\paragraph{} Now is a good time to try figuring it out on your own.
			
			\paragraph{} Looking at this, different patterns become evident. Now we can decide what is not important, and that $3(0)$ is a prime candidate. If we cast it out, we end up with:
			
			\begin{figure}[h]
				\begin{align*}
				&0 \\
				&3(0) + 2\strut^0\\
				&3(2\strut^0) + 2\strut^1\\
				&3(3(2\strut^0) + 2\strut^1) + 2\strut^2\\
				&3(3(3(2\strut^0) + 2\strut^1) + 2\strut^2) + 2\strut^3\\
				&3(3(3(3(2\strut^0) + 2\strut^1) + 2\strut^2) + 2\strut^3) + 2\strut^4\\
				&3(3(3(3(3(2\strut^0) + 2\strut^1) + 2\strut^2) + 2\strut^3) + 2\strut^4) + 2\strut^5
				\end{align*}
				
				\caption{$B$'s structure, simplified slightly}
			\end{figure}
			\newpage %aligns above figure
		
			\paragraph{} It is hard to shake the feeling that there is a pattern here. Certainly, it is intuitively visible, but maybe not mathematically so. Again, only playing with it can help. I chose to rewrite this so that the 3s are arranged in a way similar to the 2s, but without simplifying the exponents:
			
			\begin{figure}[h]
				\begin{align*}
					&0\\
					&0 + 2\strut^0\\
					&3\strut^1 + 2\strut^1\\
					&3\strut^2 + 3\strut^1 2\strut^1 + 2\strut^2\\
					&3\strut^3 + 3\strut^2 2\strut^1 + 3\strut^1 2\strut^2 + 2\strut^3\\
					&3\strut^4 + 3\strut^3 2\strut^1 + 3\strut^2 2\strut^2+ 3\strut^1 2\strut^3 + 2\strut^4\\
					&3\strut^5 + 3\strut^4 2\strut^1 + 3\strut^3 2\strut^2+ 3\strut^2 2\strut^3 + 3\strut^1 2\strut^4 + 2\strut^5
				\end{align*}
				
				\caption{$B$'s structure, simplified aggressively}
				\label{fig:bAggressive}
			\end{figure}
		
			\paragraph{} Now we can see the true underlying pattern. For those of you with moderate experience in algebra, this structure should be ringing bells. It is \textit{similar} to a binomial expansion of $(3 + 2)$, but not quite. In a binomial expansion, Pascal's Triangle is involved with the summation. However, we have no need for that here, so we can remove it from the summation notation, leaving us with the generalized form $\sum\limits_{i = 0}^s (x\strut^{s-i}y\strut^i)$. Therefore, when we plug in the pertinent values, we come up with $B_s = \sum\limits_{i = 0}^s (3\strut^{s-i}2\strut^i)$.
			
			\paragraph{} This results in the desired behavior, and has the benefit of being a non-recursive solution. It is certainly hefty, but vastly less so than the previous definition. All that is left is setting it to match the original $B$ sequence, repeating parts and all.
			
			\paragraph{} Note: Astute readers will notice that this is not a complete solution. The original $B$ sequence returns a 0 when $s$ is 0. This, however, is incapable of returning a 0. Therefore, it is a solution for all $s > 0$, rather than all $s$ values. Plugging it into $f$ would mean that $f$ is only a solution for $s > 0$ by consequence. While unfortunate, and $s$ of 0 is expected to simply return $n$, so we are not losing too much functionality. In other words, a total solution would be fantastic, but is not strictly required.
			
			\paragraph{} On topic, the now we will our newfound definition match the actual sequence's behavior. We will set $P_1$ to be the expected output values in our definition: 0, 0, 1, 1, 2, 2, 3, 3, 4, 4,.... $P_2$ will be $s$'s sequence, omitting the first 0 because our function $f$ is now no longer concerned with it: 1, 2, 3, 4, 5, 6, 7,...
			
			\paragraph{} If you would like to try and find the $P_2(s) \rightarrow P_1(s)$ function, now is a good stopping point.
			
			\paragraph{} We must find some formula $t$ which will convert $P_2 \rightarrow P_1$, as has been the standard procedure. The pattern of alternation matches $C$'s, so we can start with its definition and modify it as necessary: $t(s) = \lfloor \frac{s}{2} \rfloor$. This works for $P_2(1) \rightarrow P_1(1)$ but not $P_2(2) \rightarrow P_1(2)$. 
			
			\paragraph{} Recall that we shifted the sequence of $s$ by 1 to the left when we omitted 0. What if we try correcting for that? $t(s) = \lfloor \frac{s-1}{2} \rfloor $. This indeed works, and the reason why is sound.
			
			\paragraph{} Now we simply plug this $t$ in instead of $s$ in our definition of $B$ to find that $B_s = \sum\limits_{i=0}^{\lfloor \frac{s-1}{2} \rfloor} \left[3\strut^{\lfloor \frac{s-1}{2} \rfloor - i}2\strut^i \right]$. Next, plug this in to our $f$:
			
			\begin{figure}[h]
				\begin{equation*}
					f(n, s) = \frac{3\strut^{\lceil \frac{s}{2}	\rceil} n + 
						\sum\limits_{i=0}^
						{\lfloor \frac{s-1}{2} \rfloor} 
						\left[
							3\strut^{\lfloor \frac{s-1}{2} \rfloor - i}
							2\strut^i 
						\right]}
						{2\strut^{\lfloor \frac{s}{2} \rfloor}}
				\end{equation*}
				
				\caption{The completed $f$ function}
				\label{function:f}
			\end{figure}
		
			\paragraph{} That concludes our hunt for $f$. Now, we must ask a question: Does this function describe any number which actually exists?
			
\section{Stepping Stones}

	\subsection{Defining Existence}
	
		\subsubsection{The Question}
		
			\paragraph{} Towards the end of \ref{sss:infinite}, the utility of $f$ was discussed. To recap, diverging from the rules set by the Collatz Function by taking the even branch on an odd number will force a fraction. In our case, $f$ is treating every $n$ placed into it as if it was odd, then even, then odd, infinitely. If a number does \textit{not} actually follow this pattern, it is guaranteed to become a fraction as $s$ tends towards infinity. 
			
			\paragraph{} This phenomenon can be shown to be true because it will eventually take the even branch on an odd number. If it instead takes the odd branch on an even number, it will produce an odd number, where the next iteration will send it through the odd branch, forcing a fraction anyway.
			
			\paragraph{} We can use this to our advantage. If we were to find a number which demonstrates this odd, even, odd, even... pattern indefinitely, we have found a number which disproves the Collatz Conjecture.
			
			\paragraph{} However... finding such a number is quite difficult. It would instead make more sense to try and prove that this number \textit{cannot} exist by steadily proving that this behavior does not apply to more and more classes of integers. For example, above we proved that the lowest disproving number cannot be even. Therefore, if we were to continue looking for it, we would only need to prove things about odds. Such is what we are trying to accomplish here.
			
			\paragraph{} It all starts with a challenge:
			
			\begin{figure}[h]
				\centering
				Prove that, for any $n \geqslant 1$, there exists some value $k$ where $s \geqslant k$ results in a non-integer (fractional) output of
				\begin{equation*}
				f(n, s) = \frac{3\strut^{\lceil \frac{s}{2}	\rceil} n + 
					\sum\limits_{i=0}^
					{\lfloor \frac{s-1}{2} \rfloor} 
					\left[
					3\strut^{\lfloor \frac{s-1}{2} \rfloor - i}
					2\strut^i 
					\right]}
				{2\strut^{\lfloor \frac{s}{2} \rfloor}}
				\end{equation*}
				
				\caption{The prediction}
				\label{fig:challenge}
			\end{figure}
			\newpage %aligns above figure
			
		\subsubsection{Questioning the Question}
		
			\paragraph{} Perhaps a rewording of the challenge is in order. Pick any $n$ value, so long as it is an integer 1 or greater. Set $s$ to 1, and place these into $f(n, s)$. The prediction states that, as $s$ increases, $f(n, s)$ will eventually become a fraction. This special value of $s$ is referred to as $k$. Any $s$ less than $k$ will result in an integer output of $f(n, s)$, whereas any $s$ value greater than or equal to $k$ results in a fractional output of $f(n, s)$.
			
			\paragraph{} If this is true, it would mean that a number which exhibits the behavior odd, even, odd, even... when placed through the Collatz Conjecture does not exist, and we would have to find a new formula to describe new behavior that \textit{might} exist. In other words, start over.
			
			\paragraph{} For now, we do not know which classes of numbers this prediction is true for, although we can easily see that it must be true for at least some numbers. For example, $f(2, 1) = 7$, but $f(2, 2) = \frac{7}{2}$. This means that 2's $k$ is 2. 3's $k$ is 6:
			
			\begin{figure}[h]
				\begin{align*}
				f(3, 1) &= 10\\
				f(3, 2) &= 5\\
				f(3, 3) &= 16\\
				f(3, 4) &= 8\\
				f(3, 5) &= 25\\
				f(3, 6) &= \frac{25}{2}
				\end{align*}
				
				\caption{Finding $k$ for 3}
			\end{figure}
			\newpage %aligns figure
			
			\paragraph{} This shows that our magic number is not 2, and it is not 3, at least not under our current definition where the magic number must follow the odd, even, odd, even... pattern. Note that we will refer to this number henceforth as Epsilon ($\varepsilon$) for organizational purposes.
			
			\paragraph{} To find $\varepsilon$, we must peel away categories of numbers until only a specific subset remains. While finding $k$ for single numbers is an interesting arithmetic exercise, it is not particularly rigorous since we cannot test infinite numbers. That requires a more generalized approach...
			
			\paragraph{} Note: As we pare down potential numbers, we will update what $\varepsilon$ might be.
			
	\subsection{A Dance of Evens and Odds}
			
		\subsubsection{Defining the First Approach}
		
			\begin{figure*}[h]
				\centering
				Current state of $\varepsilon$: $\varepsilon \geqslant 1$
			\end{figure*}
		
			\paragraph{} Let us take a second to ignore all specificities of the function and recall that any time we have a fraction in the form $\frac{\text{odd number}}{\text{even number}}$, it is guaranteed to be a fraction which cannot be simplified to an integer. This is because, by definition, there are no odd numbers divisible by 2.
			
			\paragraph{} Therefore, if we can prove that the numerator will eventually become odd for any $n$ at some $s$, we have proven that the prediction is true and $\varepsilon$ does not exist. Recall that, at its heart, $f$ is 3 pieces: $\frac{An + B}{C}$. We can prove when each piece is even or odd, and use that to determine when the overall $f(n, s)$ will be even or odd.
			
			\paragraph{} $C$ is a freebie, it will always be a power of $2$, except when $s < 2$. However, this exception is of little significance: when $s < 2$, $C$ will be 1, therefore $f(n, s)$ will be an integer. Because of this, we can focus on $s \geqslant 2$, and for this interval, $C$ is indeed always even.
			
		\subsubsection{Proving An}
		
			\paragraph{} You might want to try and take a stab at it before continuing.
		
			\paragraph{} We can split $An$ into $A$ and $n$ to better consider its behavior. $A = 3\strut^{\lceil \frac{s}{2} \rceil}$, but the specifics are not as important as the general form $A = 3\strut^x$. $A$ must always be odd, because it is a power of 3, and all integer powers of 3 are odd.
			
			\paragraph{} Therefore, the answer depends on $n$. If $n$ is odd, then $An$ will be an odd $\cdot$ odd scenario, which is odd. Likewise, if $n$ is even, then we will have an odd $\cdot$ even scenario, which is even. Thus, $An$ is even when $n$ is even, and odd when $n$ is odd.
			
		\subsubsection{Proving B}
		
			\paragraph{} B is more interesting. It has no relation to $n$, instead changing with $s$. If we recall the $B$ sequence, we can see that it is 1, 5, 19, 65, 211.... This \textit{seems} to always be odd, but how can we be sure?
			
			\paragraph{} This can be approached in many ways, but it seems like another problem where analyzing the structure rather than the details would prove beneficial. Let us recall figure \ref{fig:bAggressive}, the aggressive simplification of $B$, as it shows the behavior of the summation in action:
			
			\begin{figure*}[h]
				\begin{align*}
				&0\\
				&0 + 2\strut^0\\
				&3\strut^1 + 2\strut^1\\
				&3\strut^2 + 3\strut^1 2\strut^1 + 2\strut^2\\
				&3\strut^3 + 3\strut^2 2\strut^1 + 3\strut^1 2\strut^2 + 2\strut^3\\
				&3\strut^4 + 3\strut^3 2\strut^1 + 3\strut^2 2\strut^2+ 3\strut^1 2\strut^3 + 2\strut^4\\
				&3\strut^5 + 3\strut^4 2\strut^1 + 3\strut^3 2\strut^2+ 3\strut^2 2\strut^3 + 3\strut^1 2\strut^4 + 2\strut^5
				\end{align*}
				
				\caption*{$B$'s structure, simplified aggressively (reprinted from figure \ref{fig:bAggressive})}
			\end{figure*}
		
			\paragraph{} Note that the summation is only capable of outputting the second term forward. Also note that this is a good time to stop if you are interested in solving it on your own.
			
			\paragraph{} We can check if each of the individual constituents is even or odd, looking for a pattern. Indeed there is one: Notice how the first term is always a lone 3 which gradually transforms into a lone 2. This is no coincidence -- it is precisely the behavior we defined it to have.
			
			\paragraph{} The first term will always be some power of three, while the middle terms are a product of a power of three and a power of two. The final term is always a power of two. Thus, the form it will take is:
			
			\begin{figure*}[h]
				\centering
				odd + even + ... + even
			\end{figure*}
		
			\paragraph{} What do we know about the addition of odd and even number? Well, we can strictly define the behavior. All even numbers can be rewritten as $2n$ by definition. Therefore, an even plus an even can be written as $2n + 2m = 2(n + m)$, which is also even by definition. An odd number can be rewritten, this time as $2n + 1$, so an odd plus an odd can be written as $2n + 1 + 2m + 1 = 2n + 2m + 2 = 2(n + m + 1)$, which is again even by definition. Finally, an even plus an odd is $2n + 2m + 1 = 2(n + m) + 1$, which is odd by definition.
			
			\paragraph{} Even more intuitively, an even number is just a bunch of twos, whereas an odd number is a bunch of twos and a one. Two even numbers will just be more twos. Two odd numbers will be a bunch of twos and two ones, which is just a bunch of twos. Only an even and an odd result in a bunch of twos and a one.
			
			\paragraph{} Thus, because every term in our sum is even except for the first, the summation will be odd, no matter how large $s$ becomes. $B$ is odd, always.
			
		\subsubsection{A Glimpse of K}
		
			\paragraph{} Now we can learn something about our function. Since $B$ is always odd and $An$'s odd/even status depends solely on $n$, we have two possible outcomes. The first is that $An + B$ is odd, the second is that it is even. The former occurs only when $n$ is even.
			
			\paragraph{} Recall that this is the question we are currently trying to solve. We now know that $An + B$ is odd only when $n$ is even. This tells us that $f(n, s)$ is guaranteed to be a fraction when $n$ is even. The next question is "when?"
			
			\paragraph{} It is tempting to assert that $k = 1$ for all even numbers, but a test shows this not to be the case. After all, 2's $k$ was 2. Instead, remember how $C$ is defined: $2\strut^{\lfloor \frac{s}{2} \rfloor}$. While yes, it is a power of two, it can be shown to be $2\strut^0$ when $s < 2$. Since this is our denominator, and all integers divided by 1 are also integers, then we cannot have a fraction if the denominator is 1. The denominator only changes to 2 when $s \geqslant 2$.
			
			\begin{figure*}[h]
				\centering
				Thus, for all even numbers, $k = 2$.
			\end{figure*}
		
			\paragraph{} In other words, for any even $n$, $f(n, 2)$ \textit{will} be a fraction. $\varepsilon$ cannot be an even number -- this is the first class of numbers to go. Now we only need to focus on the odds.
			
	\subsection{Doubling Down}
		
		\subsubsection{Odd Behavior}
		
			\begin{figure*}[h]
				\centering
				Current state of $\varepsilon$: $\varepsilon \geqslant 1$, $\varepsilon$ must be odd
			\end{figure*}
				
			\paragraph{} We now only need to consider what $k$ is when $n$ is odd. As usual, when stumped, we look for patterns. Our $n$ values are the infinite chain of 1, 3, 5, 7, 9, etc. What are some $k$ values for these numbers? Computers can help us calculate $k$ values quickly:
			
			\begin{figure}[h]
				\centering
				1:4, 3:6, 5:4, 7:8, 9:4, 11:6, 13:4, 15:10, 17:4, 19:6, 21:4, 23:8, 25:4, 27:6, 29:4, 31:12, 33:4, 35:6, 37:4, 39:8, 41:4, 43:6, 45:4, 47:10, 49:4, 51:6, 53:4, 55:8, 57:4, 59:6, 61:4, 63:14
				
				\caption{Pairs of $n$:$k$ values}
				\label{fig:nk}
			\end{figure}
		
			\paragraph{} There are quite a few patterns here. At the very least, something seems methodical about it. A repeating pattern for $k$ seems to be 4, 6, 4, 8, although it gets broken up with an odd 10, 12, or 14. Speaking of which, some of these have quite large $k$ values. 63's $k$ is 14, while 61's is 4. Why are they so close and yet so different?
			
			\paragraph{} Let us keep a running tally of which $n$s have the highest $k$ values. The first is 1, with a $k$ of 4. Immediately after, it is bested by 3, with a $k$ of 6. Then 7 with a $k$ of 8. Then 15 with a $k$ of 10...
			
			\paragraph{} Now hold on, these $k$ values seem to be counting up by 2. Let us take these in isolation:
			
			\begin{figure}[h]
				\centering
				1:4, 3:6, 7:8, 15:10, 31:12, 63:14
				
				\caption{The largest $k$ values at their respective points}
			\end{figure}
		
			
			\paragraph{} The $k$ values indeed seem to be going up 2 at a time. What is the significance of their respective $n$ values?
			
			\paragraph{} The $n$ values seem to be one less than a power of 2. Even better, it seems that if $n$ can be expressed as $2\strut^m - 1$, then $k = 2m + 2 = 2(m + 1)$. Note that this is merely a hypothesis, not a proof. The logical next question should be: can we prove this?
			
		\subsubsection{2's Might}
			
			\paragraph{} By asserting that $k = 2(m + 1)$ when $n = 2\strut^m - 1$, we are making a very strong statement about numbers which cannot exist. Therefore, we can plug this information into $f$ to come up with a modified form of it, we can call it $g$.
			
			\paragraph{} First, what are we plugging in? $f(2\strut^n-1, 2(n + 1))$ should always return a fraction, so let us place these values into $f$ directly to define $g$:
			
			\begin{equation*}
				g(n) = \frac{3\strut^{\lceil \frac{2(n + 1)}{2} \rceil} (2\strut^n - 2) + 
					\sum\limits_{i=0}^
					{\lfloor \frac{(2(n + 1))-1}{2} \rfloor} 
					\left[
					3\strut^{\lfloor \frac{(2(n + 1))-1}{2} \rfloor - i}
					2\strut^i 
					\right]}
				{2\strut^{\lfloor \frac{2(n + 1)}{2} \rfloor}}
			\end{equation*}
			
			\paragraph{} This can be simplified quite a bit:
				
			\begin{equation*}
				g(n) =
				\frac{3\strut^{n+1}\left(2\strut^n - 1\right) + 
				\sum\limits_{i=0}^{n} \left[3\strut^{n-i}2\strut^i\right]}
				{2\strut^{n+1}}
			\end{equation*}
			
			\paragraph{} This new definition has a notable property: because we are trying to prove that it will never be an integer, we can take the body of the definition and treat it as a number, analyzing it in isolation:
			
			\begin{figure}[h]
				\begin{equation*}
				\frac{3\strut^{n+1}\left(2\strut^n - 1\right) + 
					\sum\limits_{i=0}^{n} \left[3\strut^{n-i}2\strut^i\right]}
				{2\strut^{n+1}}
				\end{equation*}
				
				\caption{The mystery number: does it exist?}
			\end{figure}
			\newpage %aligns above figure
			
			\paragraph{} Now, we can begin the task of proving the existence of this number. Since the numerator takes the form of "an odd number times an odd number + an odd number," it will always be even. This means we cannot use our previous trick. We will need to work with more complex rules to tackle this.
			
		\subsubsection{Geometric Sequences}
			
			\paragraph{} It would be extremely useful to break apart that summation. Using the rules of summations, we can indeed simplify it a bit. Consider:
			
			\begin{equation*}
				\sum\limits_{i=0}^n 2i
			\end{equation*}
			
			\paragraph{} 
			
			
			
			
			
						
		
\end{document}